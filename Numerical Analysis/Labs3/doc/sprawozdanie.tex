\documentclass{article}
\usepackage[utf8]{inputenc}
\usepackage{polski}
\usepackage[polish]{babel}
\usepackage{bbm}
\usepackage{amsmath}
\usepackage{amsthm}
\usepackage{graphicx}
\usepackage{epstopdf}
\usepackage{float}

\newtheorem{defi}{Definicja}
\newtheorem{twr}{Twierdzenie}
\newtheorem*{dd}{Dowód}

\DeclareMathOperator{\sign}{sign}
\DeclareMathOperator{\arctg}{arctg}

\newcommand{\twopartdef}[4]
{
	\left\{
		\begin{array}{ll}
			#1 & \mbox{jeśli } #2 \\
			#3 & \mbox{jeśli } #4
		\end{array}
	\right.
}


\author{Jarosław Dzikowski 273233}
\date{Wrocław, \today}
\title{\textbf{Pracownia z analizy numerycznej} \\ Sprawozdanie do zadania \textbf{P.3.11}}
\begin{document}
\maketitle
\section{Uwagi techniczne}
Program można uruchomić normalnie z wiersza poleceń. Nie wypisuje on dużo, ponieważ wyniki doświadczeń są zamieniane na wykresy. Program ma zakomentowane wywołania funkcji produkującej dane do sporządzenia wykresu przez gnuplot'a.\\

Sprawozdanie należy kompilować z wiersza poleceń będąc wewnątrz katalogu doc.
Do skompilowania sprawozdania wymagana jest obecność folderu ,,wykresy'' z wykresami w formacie eps, które następnie będą zamieszczone w sprawozdaniu. Folder ,,wykresy'' znajduje się w folderze ,,doc''.

\section{Wstęp}
W tym zadaniu

\section{Wprowadzenie}

\subsection{Kwadratury interpolacyjne}

\subsection{Kwadratury Newtona-Cotesa}

\subsection{Twierdzenie MacLaurina}

\subsection{Metoda Romberga}
Pierdoły

\section{Doświadczenia}

\section{Wnioski}

\begin{thebibliography}{9}

\bibitem{Dahlquist&Bjorck}
Dahlquist, G., Bjo\"rck, A.
\emph{Numerical Methods in Scientific Computing, Volume I},
Society for Industrial and Applied Mathematics (September 4, 2008), 354-360.

\bibitem{Cheney&Light}
Cheney, E. W., Light, W. A.
\emph{A Course in Approximation Theory},
American Mathematical Soc., 2009, 11-22.

\bibitem{Erdos}
Erdo\"s, P.
\emph{Problems and results on the theory of interpolation, II.}
Acta Math. Acad. Sci.
Hungar. 12 (1961), 235-244.

\bibitem{Brutman}
Brutman, L.
\emph{On the Lebesgue function for polynomial interpolation.}
SIAM J. Numerical Analysis 15 (1978), 694-704.

\bibitem{Rivlin}
Rivlin, T.J.
\emph{Chebyshev Polynomials}
Wiley, New York, 1974. 2nd Edition, 1990.

\bibitem{Turetskii}
Turetskii, A. H.
\emph{The bounding of polynomials prescribed at equally distributed points.}
Proc. Pedag. Inst. Vitebsk 3 (1940), 117-127.

\bibitem{Faber}
Faber, G.
\emph{Uber die interpolatorische Darstellung stetiger Funktionen.}
Jahresber. Deutsch. Math. Verein., 23 (1914), 191-200.

\bibitem{Vertesi}
Vertesi, P.
\emph{Optimal Lebesgue constant for Lagrange interpolation.}
SIAM J. Numerical Analysis Vol. 27 (1990), 1322–1331.

\end{thebibliography}

\end{document}
