\documentclass{article}
\usepackage[utf8]{inputenc}
\usepackage{polski}
\usepackage[polish]{babel}
\usepackage{bbm}
\usepackage{amsmath}
\usepackage{amsthm}
\usepackage{graphicx}
\usepackage{epstopdf}
\usepackage[export]{adjustbox}
\usepackage{float}

\newtheorem{defi}{Definicja}
\newtheorem{twr}{Twierdzenie}
\newtheorem*{dd}{Dowód}

\DeclareMathOperator{\sign}{sign}
\DeclareMathOperator{\arctg}{arctg}

\newcommand{\twopartdef}[4]
{
	\left\{
		\begin{array}{ll}
			#1 & \mbox{jeśli } #2 \\
			#3 & \mbox{jeśli } #4
		\end{array}
	\right.
}


\author{Jarosław Dzikowski 273233}
\date{Wrocław, \today}
\title{\textbf{Bazy danych} \\ Rowery miejskie - model konceptualny} 
\begin{document}
\maketitle

\section{Diagram E-R}
Diagram E-R został zamieszczony na osobnej stronie ze względu na swoją wielkość.
\pagenumbering{gobble}% Remove page numbers (and reset to 1)
%\clearpage
\begin{figure}[p]
\centerline{	\includegraphics[width=\paperwidth, height=\paperheight, keepaspectratio]{diagram2.pdf}}
\end{figure}

\section{Komentarz}
Na diagramie widnieje osiem encji. Cztery z nich są oczywiste z punktu widzenia zagadnienia rowerów miejskich. Mamy użytkowników, rowery, stacje oraz wypożyczenia. Oczywiście użytkownicy dokonują wypożyczeń, które obejmują rowery (1 rower na wypożyczenie) oraz 2 stacje (początkową i końcową). Pozostałe cztery encje to serwisy, serwisanci, usterki oraz płatności. Rowery znajdują się albo na stacjach, albo są wypożyczone, albo znajdują się w serwisie, który też jest reprezentowany jako encja. Za wypożyczenia naliczane są płatności, które zostały przedstawione jako kolejna encja. Dodatkowo, zgłaszane są usterki, które są naprawiane przez serwisantów pracujących w serwisach.\newline
\newline
Przedstawiony model możnaby było rozbudować o następujące zależności, których nie zaznaczyłem na diagramie. Po pierwsze, serwisanci wykonują okresowe przeglądy stacji oraz rowerów. Do ich obowiązków należy także przewożenie (w nocy) rowerów z przepełnionych stacji do stacji z niedomiarem rowerów. Dodatkowo serwisanci mogą reagować na zgłaszane przez użytkowników sytuacje awaryjne (zbieranie zniszczonych przez użytkowników rowerów, naprawa stacji, etc.). Inną nieopisaną zależnością jest to, że sposób obliczania kwoty w płatności zależy od takich parametrów jak stawka za wypożyczenie konkretnego modelu roweru (Niektóre modele mogą być bardziej ekskluzywne.) lub specjalnej oferty wykupionej przez użytkownika (Np. abonament a rowery, etc.).
\newline
\newline
Nieopisane więzy są oczywiste, nie znalazłem żadnych bardziej skomplikowanych więzów:
\begin{enumerate}
	\item Czas zwrotu roweru musi być większy niż czas wypożyczenia roweru.
	\item Kwota naliczona w płatności musi być nieujemna.
	\item Data naprawy usterki musi być większa niż data jej zgłoszenia.
	\item Liczba rowerów na stacji oraz stojaków musi być nieujemna.
	\item Data pokrycia płatności musi być większa niż data jej wystawienia.
\end{enumerate}

\section{Opis ról}
\subsection{Rola użytkownika}
\begin{enumerate}
	\item Może dodawać nowe wypożyczenia.
	\item Prawo do wglądu do historii wypożyczeń.
	\item Prawo do wglądu do historii płatności.
	\item Może sprawdzać stan stacji (Czy są w niej jakieś rowery, etc.).
\end{enumerate}

\subsection{Rola serwisanta}
\begin{enumerate}
	\item Może dodawać nowe usterki.
	\item Prawo do wglądu do historii usterek.
	\item Może naprawiać usterki.
	\item Może przewozić rowery.
	\item Może dokonywać kontroli stanu stacji.
	\item Prawo do odczytu i modyfikacji zawartości serwisu, w którym pracuje.
\end{enumerate}

\subsection{Rola księgowego/policjanta}
\begin{enumerate}
	\item Ma prawo do modyfikacji oraz odczytu danych płatności.
	\item Ma prawo do odczytu historii wypożyczeń.
	\item Ma prawo do odczytu danych użytkowników.
	\item Nakłada płatności na użytkowników.
	\item Znajduje wypożyczenia z brakiem zwrotu roweru i nakłada kary na użytkowników wypożyczających.
\end{enumerate}

\end{document}
